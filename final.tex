\documentclass{article} % For LaTeX2e
\usepackage[usenames,dvipsnames,svgnames,table]{xcolor}
\usepackage{nips12submit_e,times}
\usepackage{amsmath}
\usepackage{amssymb}
\usepackage{url}
\usepackage{color}
%\documentstyle[nips12submit_09,times,art10]{article} % For LaTeX 2.09

\title{Structured Statistical Source Code Prediction\\
{\small CMU 10-701 (Machine Learning) - Final Project Report}}
\author{
  Cyrus Omar\\
  \texttt{comar@cs.cmu.edu}
  \And
  Salil Joshi\\
  \texttt{salilj@cs.cmu.edu}
  \And Fl\'avio Cruz\\
  \texttt{fmfernan@cs.cmu.edu}
}

\newcommand{\fix}{\marginpar{FIX}}
\newcommand{\new}{\marginpar{NEW}}

\usepackage{xypic}

\nipsfinalcopy % Uncomment for camera-ready version

\begin{document}


\maketitle

\section*{Introduction}
Programming languages are formal systems with rich syntactic and semantic structure. They are
also human systems, in that they are used extensively by people in patterned ways to express their
intent. Many tools are designed to help people write code more efficiently by predicting
the source code that a developer intends. For example, code completion systems for editors like Eclipse for Java display pop-up menus containing the members of to the class of the variable being manipulated. 

Typical code completion systems use only this sort of semantic
information about the language and the libraries being used, and do not incorporate data about how developers have written programs
in the past. Recent work by Hindle et al.~\cite{Hindle:2012:NS:2337223.2337322} demonstrated, however, that source code could be successfully predicted statistically
using a simple $n$-gram model that used a tokenized, rather than structural, representation of source code. 

Our project aims to unite the structured and statistical approaches to source code prediction.
That is, rather than using a tokenized representation of source code, we would like to do
statistical prediction on a more natural representation of source code: the typed syntax tree.  We can then condition our predictions using semantic information, specifically:
\begin{itemize}
\item the \emph{type}, denoted $\tau$, of the expression being predicted (e.g. \verb|int| or \verb|Color|)
\item the \emph{syntactic context}, denoted $\sigma$, in which the expression occurs (e.g. whether the expression is an argument of a function call, the guard of an \verb|if| statement, etc.)
\item the \emph{program context}, denoted $\Gamma$, in which the expression occurs (i.e. the set of variables paired with their types that are in scope at the location that the expression occurs.)
\end{itemize}

For example, if a user has entered the code \texttt{Planet destination = }, where \verb|Planet| is an enumeration containing \verb|Mercury|, \verb|Venus|, \verb|Earth|, etc. (but not \verb|Pluto|, of course), then we have that the expected type at the cursor is \verb|Planet|, the syntactic context is \emph{assignment}, and given a program context, our prediction space should only assign non-zero probabilities to:
\begin{itemize}
\item literal members of the \verb|Planet| enumeration (e.g. \verb|Planet.EARTH|)
\item variables and fields of type \verb|Planet| available from the program context
\item calls to methods available from the program context that have return type \texttt{Planet}\footnote{We can consider  operators like $+$ and $[]$ as methods of the built-in types in Java.}
\end{itemize}
To assign a probability to an expression, denoted $e$, we first determine how likely it is that the expression is of each of these three \emph{syntactic forms}, where syntactic forms are denoted $\phi$. For each form, we can then assign probabilities to particular expressions of that form according to some form-specific conditional distribution. This model can be understood as a graphical model, as shown in Figure \ref{graphicalmodel}, where the syntactic form is a latent variable. The conditional distributions for both the syntactic form and expression are learned using data gathered from analyses of prior code corpuses (smoothed using some suitable method in cases where enough information is not available).

\begin{figure}
    \begin{displaymath}
    \xymatrix{
      *+[F]{\color{ForestGreen}\txt{Type ($\tau$)}} \ar[dr] \ar[drr] & & \\
      & *+[F]{\color{black} \txt{Syntactic\\Form ($\phi$)}} \ar[r] & *+[F]{\color{cyan}\txt{Expression ($e$)}}  \\
      *+[F]{\color{ForestGreen}\txt{Syntactic\\Context ($\sigma$)}} \ar[ur] \ar[urr] & & *+[F]{\color{ForestGreen}\txt{Program\\Context ($\Gamma$)}} \ar[u]  
    }
    \end{displaymath}
\caption{A graphical model representing our approach. Green variables are always observed (we do not assign marginal distributions to them.) The syntactic form, $\phi$, is a latent variable, and the expression, $e$, is unknown. Note that the syntactic form is a function of the expression. \label{graphicalmodel}}
\end{figure}

In addition to the applications to code completion systems in code editors like Eclipse, more accurate source code prediction techniques could be useful for other programming tools. For example, programmers with severe physical impairments may benefit from predictive programming interfaces that allow them convey source code using devices more limited than a keyboard, incorporating the predictions to reduce the amount of information that needs to be conveyed explicitly by the user.

\section*{Related Work}
As mentioned above, work by Hindle et al. \cite{Hindle:2012:NS:2337223.2337322} showed that using a simple $n$-gram model of source code, represented as sequences of tokens, produced reasonably accurate predictions. This work (published this summer) suggested improving this model with structural and semantic information as future work. We have corresponded with the authors of this paper.

Work examining the ``uniqueness'' of software has found that source code can often be decomposed into commonly-seen motifs \cite{Gabel:2010:FSE}, further suggesting that prediction that uses structural information may be a particularly helpful approach in the domain of source code.

Code completion systems that use statistical information in a limited context, such as specifically for predicting which methods of a class will be called, have been proposed in the literature \cite{Bruch:2009:LEI:1595696.1595728,robbes_how_2008}. Other systems, such as Calcite \cite{mooty_calcite:_2010}, have produced contextually-relevant {\em code snippets} or examples drawn from a code corpus or web search, where more frequently encountered snippets are preferentially shown. These can all be construed as related, more specialized, variants of the approach we are taking in our project.


\section*{Method}

\subsection*{Model}
The model we use for code prediction can be stated as:
$$ e = \arg\max_e P(e ~|~ \texttt{type} , \texttt{context} ) $$

Here \texttt{type} is the expected type of the expression. If a given expression $e$ does not have the required type, then $P(e ~|~ \texttt{type})$ is necessarily $0$ (since $e$ cannot appear in this position in any well-typed programming). This drastically reduces our search space.

\texttt{context} is the program context in which this expression is to appear. Some examples of contexts:
\begin{itemize}
  \item Assignment: \texttt{Planet p =} $\Box$
  \item If-test: $\texttt{if }\Box\texttt{ then}\ldots\texttt{else}\ldots$
  \item Method arguments: $\texttt{O.foo }\Box$
\end{itemize}

The $e$ in the equation above ranges over possible expressions. However we do not search over \emph{all} possible expressions. This is impossible since there may be an infinite number of well formed expressions. Moreover, it is unneccessary since we need only predict the \emph{root} of the syntax tree for $e$. The rest of the nodes in the tree can then be predicted recursively.

Thus we first identify the three forms that the tree $e$ can take:
\begin{itemize}
  \item $e$ may be just a literal of the appropriate type. For example if the type is an \texttt{enum}, $e$ may be a member of that \texttt{enum}
  \item $e$ may be a variable in scope of the appropriate type.
  \item $e$ may be a call to a method that returns the appropriate type. In this case, we predict only the method itself. The arguments to the method can be predicted later recursively.
\end{itemize}

\subsection*{Analysis}
The parameters to our model then, are the probabilities of occurrences of each of these three forms (given the type and context), and in particular the probabilities of occurences of particular expressions of each form. Thus while analysizing a code corpus, we calculate the following:
\begin{itemize}
  \item Probability that $e$ is a literal: $P(e \text{ is a literal} ~|~  \texttt{type} , \texttt{context} ) $
    \begin{itemize}
      \item Probability that $e$ is a specific literal $l$, given that it is \emph{some} literal: \\$P(e = l ~|~ e \text{ is a literal},  \texttt{type} , \texttt{context} )$
    \end{itemize}
  \item Probability that $e$ is some (any) variable $x$ in scope: $P(e = x ~|~ \texttt{type} , \texttt{context})$
  \item Probability that $e$ is a method call: $P(e \text{ is a method call} ~|~  \texttt{type} , \texttt{context} ) $
    \begin{itemize}
      \item Probability that $e$ is a call to a specific method $f$, given that it is \emph{some} method call: \\$P(e = f \ldots ~|~ e \text{ is a method call},  \texttt{type} , \texttt{context} )$
    \end{itemize}
\end{itemize}

\subsection*{Prediction}

Since the number of possible contexts is small and finite, we will always have statistical data about expressions in any given context. However, this is not the case for types, since a programmer may define types that we have never seen before. If this situation occurs, we need to condition probabilities of expressions on just the context and not the type. 
$$ e = \arg\max_e P(e ~|~ \texttt{context} ) = \arg\max_e \Sigma_{\texttt{type}} P(e ~|~ \texttt{type} , \texttt{context} ) $$

Then the \emph{form} of $e$ can be predicted using this equation, however it will not give us a useful prediction for the specific expression (since, if we have not seen the expected type before, we cannot have seen expressions of that type before). So when making a prediction, we use a two-step method: First predict the most likely form of the predicted expression, then either predict or try to guess the specific expression. There are three possibilities for the form of $e$:
\begin{description}
  \item[Literal:] If we have data available about probabilities of literals of the expected type, then we may use that to make a prediction. However if we do not, then we predict \emph{any} literal of the expected type.
  \item[Variable:] Here there are two possibilities. We may either predict the most \emph{recently} used variable in scope, or the most \emph{frequently} used variable in scope. We need further analysis and experimentation to decide which of these two possibilities gives more useful results.
  \item[Method Call:] Similar to the case for literals. If we have data available about probabilities of methods that return the expected type, then we may use that to make a prediction. However if we do not, then we predict any method that returns the expected type.
\end{description}

This two-step approach negates the need to solve the difficult problem of smoothing our data. Smoothing at the level of contexts is unneccessary since we always have enough data. Smoothing at the level of types is also not required, since the prediction algorithm can handle cases where we have insufficient data in a natural fashion.

\subsection*{Implementation}
We use the Java programming language to perform experiments using our methodology. The ideas underlying our project are applicable to any statically typed programming language, but Java has some advantages that are are relevant to us. First, there are large Java code corpuses freely available. Second, there is prior work on predicting Java programs that we can use as a basis for evaluating our model (see above). Lastly, we can use the Eclipse IDE which gives us access to a powerful toolchain for working with Java programs.

We implemented our prediction model using the Eclipse Java Development Tools (JDT) library. This library provides not only facilities for parsing Java code and retrieving the type information of code expressions, but also for doing AST visits of every expression and retrieving information about types, such as methods, fields, argument types, variables in scope, etc.

To learn the parameters of our model, we determine the frequencies of each expression form, given its type and also it's context.
Context can be either \textit{variable declaration} (expression appears as the initializer of a variable), \textit{statement}, \textit{method argument} (expression is the argument of a method invocation) or \textit{other} (everything else).

We have implemented three code visitors to collect data about the three expression forms:

\begin{description}
   \item[Literal Visitor:] For primitive types such as integers, floats, doubles or strings we count the frequency of each literal in the code. For enumeration types (like \texttt{Planet}), we count the frequency of each element of the enumeration.

   \item[Variable Visitor:] For each type, we count the number of times a variable of this type was referenced in the code.

   \item[Method Visitor:] We group methods by the return type, so that we can count the number of times each method is used. We also collect data about all the declared methods.
\end{description}

For prediction, we make use of another visitor in order to go through all the expressions in the target file. To compute the probability of an expression $e$ to be found at that point in the file, we determine the context $\sigma$ of $e$ and it's type $\tau$. We
then count the number of literals, variables and methods that we found during the collection phase with the same $\sigma$ and $\tau$. We then do a case analysis on the form of $e$:

\begin{itemize}
   \item For literals, we compute the relative frequency of all literals over all expressions forms, and multiply it by the relative frequency of the literal value in relation to all literals with the same type and context.
   \item For variables we also compute the relative frequency of variables over all expressions forms and divide this by the number of variables of the same type in the scope of $e$. To find all variables, we first find the method $m$ where $e$ is located and then do a visit on $m$ to locate all variables that appear before $e$ and have type $\tau$.
   \item For method invocations, we compute the relative frequency of methods over all expression forms and multiply it by the relative frequency of methods of type $\tau$ and context $\sigma$. 
\end{itemize}

\section*{Experiments}

Although the best method to test the quality of our models would be to perform an user study, we are a bit more limited in time and resources. Therefore, we used already available software code to automatically test our prediction facilities.

Our experiments makes use of freely available code corpuses listed in Qualitas Corpus (\url{http://qualitascorpus.com/}).
Table~\ref{tab:projects} presents the projects we picked from the catalogue. We also show the number of files and
lines of code (LOC) of every project.

\begin{table}[ht]
\centering
\begin{tabular}{|l|c|c|l|}
\hline
\textbf{Project} & \textbf{\# Files} & \textbf{\# LOC} & \textbf{Description} \\
\hline
Ant & 1196 & 256039 & a software tool for automating software build processes \\
ANTLR & 394 & 66845 & a framework for implementing compilers \\
Art Of Illusion & 487 & 153645 & an open source 3D modelling and rendering studio \\
Axion & 237 & 42329 & a relational database system supporting SQL and JDBC \\
Batik & 1664 & 366506 & a library for manipulation of SVG files \\
FindBugs & 1149 & 207827 & a program that uses static analysis to look for bugs \\
JFreeChart & 1005 & 321551 & a library for displaying professional charts \\
\hline
\end{tabular}
\caption{Software projects used in our experiments.}
\label{tab:projects}
\end{table}

For each project, we consider a subset of files for testing, the test set, and all the remaining files as the training set.
In our experiments we want to determine the probability of our model to predict the expressions found in the test set (the prediction rate), given the parameters we found by analyzing the training set.

To get a more accurate prediction rate, we use a ten-fold cross validation, where we select 10\% of the files at random as the test set and build our model by using the remaining 90\% of the files. For each file in the test set, we visit all the expressions (including sub-expressions) and compute the prediction rate. We then average the prediction rate of all expressions found to produce the prediction rate for that file. Finally, we average the prediction rate of all files to get a prediction rate for the current iteration of the cross validation procedure.

For comparision with the state-of-the-art, we generated probabilities using an $n$-gram model~\cite{Hindle:2012:NS:2337223.2337322}, with $n = 3$ tokens. The $n$-grams were generated using a version of the CMU Language Modeling Toolkit and the probabilities were computed using several Python scripts.

\section*{Conclusion}

Our project attempts aims to develop prediction models for source code completion through the use of type and context information. We intend to apply them to code software corpuses in order to get assess their accuracy. Finally, we also want to compare our results with other code completion prediction models.


\bibliography{midway}
\bibliographystyle{plain}

\end{document}
